The work plan for my thesis consists of five major steps, that can be summarized as follows:

\begin{enumerate}
\setcounter{enumi}{-1}
\item Study the existing message passing systems, event based programming models and the example network processing units
\item Build a simulation model of the system
\item Measure the needed attributes from the example hardware and software
\item Plug in the attributes to the simulation model
\item Run simulations
\item Validate that the simulations correspond to the hardware system
\end{enumerate}

In step 0 we have studied the OpenEM, MPI, future-promise constructs, and the example hardware to be able to create a simulation model and measure the needed parameters for it. We are currently working on step 1, during which we will build a simulation model with PSE. The high level model has already been built, however the packet scheduling and applications still need to be enhanced.

In step 2, we will instrument and measure the needed attributes from the example hardware system. Main interest will be on the communication and memory latencies. We will begin the measurements right after we overcome the packet dropping and deadlocking problems of the system.

Then, in step 3, the measurement attributes will be plugged in the simulation model. Finally, the simulations will be run and further tests are run to validate correct behaviour of the simulation model. The experiment will be reiterated while need be and the resources are sufficient, starting from step 1.

%%% Local Variables:
%%% mode: latex
%%% TeX-master: "thesis-plan-hartikainen"
%%% End:
