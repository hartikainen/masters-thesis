In this thesis, we hope to understand the applicability of network processing units and event driven programming models in distributed stream computation context. We believe that, by distributing the computation and using task based parallel programming models, both the programmability and the efficiency of such systems can be improved. We hope understand the load-balancing options, that is, mainly the communication latencies and mechanisms, of such systems.

We will measure the performance of a multi-unit network processing system, and build a simulation model based on those measurements. Especially interesting are the delays in the processing pipeline and memory latencies of the system. The simulation model is used to further understand limitations of such a system and its suitability as stream computing platform.

The modeling will be done using Performance Simulation Environment (PSE), based on the values gathered from various measurements. The measurement setup consists of eight 32 MIPS core network processing units, connected to each other over an ethernet switch.

The thesis is done under the Embedded Systems Group in Aalto University with connections to the ParallaX research project. The actual starting point for this thesis is defined by the recent bachelor's theses (Kiljunen, Teemaa) and master's theses (Hanhirova, Saarinen, Rasa), written in the same group.

%%% Local Variables:
%%% mode: latex
%%% TeX-master: "thesis-plan-hartikainen"
%%% End:
