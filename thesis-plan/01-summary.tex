In this thesis, we hope to understand the applicability of NPU devices and event driven programming models in stream computation. We believe that by distributing the computation, utilizing external accelerators and task based programming models, both the programmability and the efficiency of the devices, can be improved.

We will measure the performance of a multi-node network processing system, and create a simulation model based on those measurements. The simulation model is used to further understand limitations of such a system and its suitability as stream computing platform.

The modeling will be done using Performance Simulation Environment (PSE), based on the values gathered from various measurements. The measurement setup consists of 8 Cavium Octeon II, 32 MIPS core blades, connected to each other over an ethernet switch. The emphasis of the measurements will be on the communication delays between the blades.

The thesis is done under the Embedded Systems Group in Aalto University with connections to the ParallaX research project. The actual starting point for this thesis is defined by the recent bachelor's theses (Kiljunen, Teemaa) and master's theses (Hanhirova, Saarinen, Rasa), written in the same group.

%%% Local Variables:
%%% mode: latex
%%% TeX-master: "thesis-plan-hartikainen"
%%% End:
