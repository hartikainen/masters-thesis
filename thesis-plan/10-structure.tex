The rough of the thesis structure is presented below. The bullets under the subsections are general ideas of what should be included in each chapter, not subsubsections.

\begin{itemize}
\item Abstract
\item Preface
\item Contents
\item Abbreviations
\end{itemize}

\begin{enumerate}
\item Introduction
  \subitem Research Problem
  \begin{itemize}[leftmargin=45px]
    \item Distributing stream computing, how to load-balance the computation across the nodes?
    \item How long delays occur if the computation is moved from node to another?
  \end{itemize}
  \subitem Contributions
  \subitem Structure

\item Background
  \subitem Parallel Computing
  \begin{itemize}[leftmargin=45px]
    \item Free lunch is over. Why do we parallelize computing in the first place?
    \item MPSoC devices
    \item From threads to tasks. Why do we need new parallel programming methods such as OpenEM?
  \end{itemize}
  \subitem Stream Computing
  \begin{itemize}[leftmargin=45px]
    \item Internet of Things, fog computing.
    \item From Von Neumann to Streams. Window-type view of the data. Real-time.
  \end{itemize}
  \subitem Distributed Computing
  \begin{itemize}[leftmargin=45px]
    \item Distributed computing is widely used e.g. in scientific computing
    \item Message passing
  \end{itemize}
  \subitem I/O Virtualization
  \begin{itemize}[leftmargin=45px]
    \item ??
  \end{itemize}

\item Distributed, Parallel Stream Computing
  \subitem Open Event-Machine
  \begin{itemize}[leftmargin=45px]
    \item Event driven parallel programming model
    \item Currently no support for distributed computation
  \end{itemize}
  \subitem The context of computation
  \begin{itemize}[leftmargin=45px]
    \item What is actually needed to move the computation between the nodes?
  \end{itemize}
  \subitem Hardware requirements
  \begin{itemize}[leftmargin=45px]
    \item What is required from the hardware for the distributed computing to be feasible
    \item Especially input and output latencies are important
  \end{itemize}

\item Measuring the system
  \subitem Measurement Setup
  \begin{itemize}[leftmargin=45px]
    \item NPU nodes
    \item Instrumenting the applications.
  \end{itemize}
  \subitem Communication Latencies
  \begin{itemize}[leftmargin=45px]
    \item Statistical behaviour of input and output under varying workload
    \item Simple tests with message passing. MPI or future-promise
  \end{itemize}
  \subitem Memory Latencies
  \subitem Results

\item Simulation Model
  \subitem PSE - Performance Simulation Environment
  \subitem Workload model
  \subitem Hardware model
  \begin{itemize}[leftmargin=45px]
    \item Scheduler, input/output, NPU nodes
  \end{itemize}
  \subitem Software model
  \begin{itemize}[leftmargin=45px]
    \item Scheduler, OpenEM-type application model
  \end{itemize}

\item Discussion
  \subitem Challenge
  \subitem Discoveries
  \subitem Related Work
  \subitem Future Work

\item Conclusions
\end{enumerate}

%%% Local Variables:
%%% mode: latex
%%% TeX-master: "thesis-plan-hartikainen"
%%% End:
