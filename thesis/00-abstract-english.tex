\thesisabstract{english}{
This thesis investigates the use of measurement, simulation, and modeling methods for the performance analysis of packet processing systems, and more precisely hardware accelerated multiprocessor system-on-chip (MPSoC) devices. To guarantee the tight latency and throughput requirements, the devices often incorporate complex hardware accelerated packet scheduling mechanisms. At the same time, due to the complexity of these systems, different software abstractions, such as task-based programming models, are required to make the development of the packet processing applications. Performance analysis of hardware accelerated packet processing systems running on task-based parallel programs is not straightforward.

We demonstrate that, with extended queue disciplines and support for modeling parallelism, resource network methodology is a viable approach for modeling complex multiprocessor system-on-chip based NPUs. The main contributions of our work are three-fold. First, we have extended the toolset of an existing in-house modeling and simulation software, Performance Simulation Environment. The extensions enable modeling of user-definable queue disciplines To enable flexible modeling of complex hardware interactions of MPSoCs and the parallelism of task-based programming models. Secondly, we have studied, instrumented, and measured the characteristics of a reference MPSoC system. Finally we have modeled a multi-blade network processing system with customizable workload and task-parallel application models, and run simulation experiments.

In both experiments, the model acts as expected. According to the experiment results, the resource network concept seems to be a viable tool for the performance analysis of complex hardware and software co-scheduled manycore packet processing systems. The chosen abstraction level provides good balance between the functionality, ease of use, and simulation performance.

}

%%% Local Variables:
%%% mode: latex
%%% TeX-master: "thesis-hartikainen"
%%% End:
