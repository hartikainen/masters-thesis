\thesisabstract{english}{
This thesis investigates the use of measurement, simulation, and modeling methods for the performance analysis of packet processing systems, and more precisely hardware accelerated multiprocessor system-on-chip (MPSoC) devices running task-parallel applications. To guarantee the tight latency and throughput requirements, the devices often incorporate complex hardware accelerated packet scheduling mechanisms. At the same time, due to the complexity of these systems, different software abstractions, such as task-based programming models, are used to develop packet processing applications. Also, the characteristics of the packet streams are often unknown in advance, making the performance analysis of these systems is non-trivial.

We demonstrate that, with extended queue disciplines and support for modeling parallelism, resource network methodology is a viable approach for modeling complex MPSoC based systems running task-based parallel applications on dynamic workloads. The main contributions of our work are three-fold. First, we have extended the toolset of an existing in-house modeling and simulation software, Performance Simulation Environment. The extensions enable modeling of user-definable queue disciplines which further enables flexible modeling of complex hardware interactions of MPSoCs and the parallelism of task-based programming models. Secondly, we have studied, instrumented, and measured the characteristics of a packet processing system. Finally we have modeled a multi-blade packet processing system with customizable workload and task-parallel application models, and run simulation experiments.

In both experiments, the model acts as expected. According to the experiment results, the resource network concept seems to be a viable tool for the performance analysis of complex hardware and software co-scheduled manycore packet processing systems under dynamic workloads. The chosen abstraction level provides desired balance between the functionality, ease of use, and simulation performance.
}

%%% Local Variables:
%%% mode: latex
%%% TeX-master: "thesis-hartikainen"
%%% End:
