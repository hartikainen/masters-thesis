\thesisabstract{english}{
The size of the digital universe is estimated to grow ten-fold, from 4.4 zettabytes to 44 zettabytes, between 2013 and 2020. At the same time, the silicon industry has faced the so-called powerwall, and the performance requirements of computing are ever higher. Cloud and fog computing promise an attractive option for the IT industry, to offload computation from end-devices to centralized datacenters. This introduces new challenges for the communication networks, and especially the packet processing systems.

In this thesis, we investigate the use of measurement, simulation, and modeling methods for the performance analysis of packet processing systems, and more precisely Network Processing Units (NPU). The packet scheduling mechanisms of the NPU devices play the key role for guaranteeing the latency and throughput requirements. The packet ordering requirements, together with parallelized packet processing, require careful design of the simulator software to mimic the correct behavior of the system under study.

% Our focus is on, how to efficiently model and simulate the scheduling algorithms on high abstraction level, while maintaining the meaningful behavior at the same time.

We show that, with extended queue disciplines and support for modeling parallelism, resource network methodology is a viable approach for modeling complex multiprocessor system-on-chip based NPUs. The main contributions of our work are three-fold. First, we have extended the toolset of existing in-house simulator software, Performance Simulation Environment, to enable modeling of customized queue disciplines needed to model the complex system interactions of the MPSoCs. Secondly, we have studied, and instrumented and measured the characteristics of a customized modern NPU. Lastly, we have built a simulation model based on the reference NPU and ran experiments to validate the proper model and simulator behavior.

The experiment results are encouraging. In both experiments, the model acts as expected. According to the experiment results, the resource networks concept seems to be a viable tool for the performance analysis of complex hardware and software co-scheduled manycore packet processing systems. In addition to its efficiency and accuracy, Performance Simulation Environment, provides an intuitive approach to modeling, by logically decoupling the hardware and software models.

% The new features of the simulator were presented for the industry partners as a part of ParallaX group's fall assembly, and especially the new scheduling mechanisms gained major interest amongst the participants. The toolset has also been used both by the industry and in multi-university organized teaching.
}

%%% Local Variables:
%%% mode: latex
%%% TeX-master: "thesis-hartikainen"
%%% End:
