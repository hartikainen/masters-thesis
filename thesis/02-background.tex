\chapter{Background}
\label{chapter:background}

\todo[inline]{write introduction to the backgroup chapter}
\todo[inline]{rename the background}

\section{Fog Computing}
\label{section:fog-computing}
% % http://www.datacenterknowledge.com/archives/2013/08/23/welcome-to-the-fog-a-new-type-of-distributed-computing/
% % Fog Computing and Its Role in the Internet of Things [http://conferences.sigcomm.org/sigcomm/2012/paper/mcc/p13.pdf]
% % https://techradar.cisco.com/technology/fog-computing#prettyPhoto

% fog computing is a paradigm that extends cloud computing to the edge of the network
% in fog computing, intelligent nodes are deployed at strategic locations of the network between the centers and the cloud
% fog computing is similar to the cloud computing, but closer to the end user, the edge, or the ground.
% scalability, reliability, faster response time, reduced cost

% fc enables capabilities that are not possible otherwise
% trains, plane, oil rig generate huge amount of data
% wireless network dont have bandwidth
% too expensive to deliver the data
% sending packet from end user to cloud might be too slow
% manufacturing: decision need to be made in fractions of seconds
% bring processing to the data, not vice versa

% improved reliability:
% network connection to the cloud goes down -> decisions can still be made
% fog computing does not contradict cloud computing, nor replace it, it complements it with much needed compute, storage and networking capabilities, at stratetig loc between sensors and data
% real-time analytics, machine learning algorithms
% analytics and ml algorithms are essential ingredients in many fog computing applications
% data into actionable insights and BI

% huge business opportunity

% % Finding your Way in the Fog: Towards a Comprehensive Definition of Fog Computing [http://delivery.acm.org/10.1145/2680000/2677052/p27-vaquero.pdf?ip=130.233.87.116&id=2677052&acc=ACTIVE%20SERVICE&key=74A0E95D84AAE420%2E1D2200B0A299939C%2E4D4702B0C3E38B35%2E4D4702B0C3E38B35&CFID=562969494&CFTOKEN=64137578&__acm__=1448197233_c568466fa204a88d9106d97b20594a8d]
% A definition for fog
% challenges: privacy, volatility, bandwidth, mobility, billions of devices
% According to [https://www.cisco.com/web/about/ac79/docs/innov/IoT_IBSG_0411FINAL.pdf], there will be 50 billion devices connected to the Internet in year 2020.
% This growth is explained by not only the growth in number of mobile devices (mobile phones and tablets, especially in developing countries)
% +!! the proliferation of IoT devices and sensors. (not included in the 50 billion)

% ``Configuring and keeping updated and secure fog networks,
% services and devices is done separately (e.g. routers, servers,
% services and devices are separately managed by different
% people). These tasks are labour intensive and error prone.
% For instance, well-known Internet companies claim a single
% admin handles thousands of machines running a single service
% type. Configuring and maintaining many different types
% of services running on billions of heterogeneous devices will
% only exacerbate our current management problems. The
% fog needs heterogeneous devices and their running services
% to be handled in a more homogeneous manner; ideally fully
% automated by software.
% Network Function Virtualisation (NFV) is arguably the
% most remarkable technology in this regard. NFV is the reaction
% of telco operators to their lack of agility and constant
% need for reliable infrastructures. NFV tries to provide
% the ability of dynamically deploying on-demand network services
% (e.g. a firewall, a router or a WAN accelerator, a new
% LAN or a VPN) or user-services (e.g. a database) where
% and when needed. Software Defined Networks SDN are one
% of the pillars needed for NFV, since some network services
% (e.g. creating new virtual networks on top of the physical
% infrastructure) can be done by software only. For instance,
% some gateways can be deployed as virtual machines and their
% traffic can be tightly controlled thanks to SDN capabilities
% in a local edge cloud, see Figure 1. The softwareisation of
% a classically hardware-driven business built around routers
% and servers where services got deployed will result in cheaper
% and more agile operations.
% A complementary approximation is proposed by Cisco
% with its first software-only version of the IOS wrapped in
% with a Linux distribution (IOx)3
% . The router itself becomes
% an SDN-enabled virtualisation infrastructure where
% NFV and application services are deployed close to the place
% where they are actually going to be used. But IO computing
% capabilities will still be limited (edge routers are not
% carrier grade after all).''

% according to [http://www.tmcnet.com/tmc/whitepapers/documents/whitepapers/2013/9378-bell-labs-metro-network-traffic-growth-an-architecture.pdf]4G LITE/EPC is supposed to be ready in 2017. It will expand the bandwidth of edge networks.


% % Overview of fog: [http://annals-csis.org/Volume_2/pliks/503.pdf]
% % [http://csjournals.com/IJITKM/PDF%208-2/13.%20Sagar.pdf]

% % Power aware load balancing for cloud computing [http://www.iaeng.org/publication/WCECS2011/WCECS2011_pp127-132.pdf]
% % hyvaa kamaa? [Raghava & Singh: Comparative Study on Load Balancing Techniques in Cloud Computing]

% % Power Aware Load Balancing for Cloud Computing [http://www.iaeng.org/publication/WCECS2011/WCECS2011_pp127-132.pdf]

\section{Parallel Computing}
\label{section:parallel-computing}

\todo[inline]{Free lunch is over. Why do we paralellize computing in the first place?}
\todo[inline]{MPSoC devices}
\todo[inline]{From threads to tasks. Why do we need new parallel programming methods such as OpenEM?}

\section{Dataplane}
\label{sec:dataplane}

\todo[inline]{clean API with a set of coherent libraries and drivers}

\todo[inline]{generic support for many CPUs/NICs}
\todo[inline]{easy to use and understand}
\todo[inline]{documentation}

\todo[inline]{DPDK subscribers/commits}

\todo[inline]{Many of today's dataplane architectures use run to completion model. No cycles can be wasted for scheduling. Applications poll for received packets. Power management?}

\todo[inline]{poll mode drivers}

\todo[inline]{DPDK almost ISA neutral, but has to be done properly}

\todo[inline]{DPDK: Hugepages -> less TLB cache misses}

\todo[inline]{Legacy dataplane? Legacy IPsec?}

\todo[inline]{Protocols, addresses, ports, packet dropping, flow reordering}
\todo[inline]{Reference implementations: Intel DPDK, ODP}

\section{Open Event-Machine}

\todo[inline]{vs. Intel DPDK, ODP??}
\todo[inline]{Mapping between fastpath hardware and dataplane application}
\todo[inline]{Event driven parallel programming model}
\todo[inline]{Currently no support for distributed computation}

\section{Stream Computing}

\todo[inline]{Internet of Things, fog computing}
\todo[inline]{From Von Neumann to Streams. Window-type view of the data. Real-time.}

\section{Distributed Computing}

\todo[inline]{Distributed computing is widely used e.g. in scientific computing}

%%% Local Variables:
%%% mode: latex
%%% TeX-master: "thesis-hartikainen"
%%% End:
