\chapter{Performance Simulation Environment}
\label{chapter:pse}

\fixme{banks et al. simulation stuff?}

This chapter describes Performance Simulation Environment (PSE) and its suitability for performance analysis of a stream computing system. We begin with an overview of PSE tools and concepts, and continue by describing the three components that describe a PSE model. After that, we go through the simulation and monitoring of PSE applications. Finally, we address the problems and deficiencies that needed to be resolved before PSE could be used for performance analysis of a stream computing system.
\fixme{correct this}

\section{Overview}

\fixme{TODO:
  \begin{enumerate}
  \item What can be modeled?? What can't be?
  \item Reusability
  \item Different simulation techniques: discrete, continuous, parallel
  \item Different modeling mechanisms: functional vs. performance etc...
  \item describe the simulation type somewhere. e.g. Discrete event simulation
  \item Model represents a system.
  \item In PSE, the simulation model is created by graphical user interfaces. Internally the model is presented as a simple text format that is then parsed by tgc.
  \item reusable components
  \item stochastic model?
  \end{enumerate}
}



PSE is a toolset for performance analysis of -- initially designed, but not limited to -- parallel computing systems. The tools can be divided into graphical model editors, compiler tools, and runtime libraries. Their usage is best described together with the typical modeling workflow: first the model editors are used to build the PSE model representation of a system; then the compilers are used to generate C-code of the model description; finally the generated C-code is compiled together with the runtime libraries into an executable simulation program. The resulting program runs on top of Linux operating system on commodity hardware.

Rsim: More detailed hardware simulation, with focus on ILP effects of processors
      Detailed shared-memory multiprocessor models
      They show that the previous simple-processor-models cannot demonstrate the complicated dynamically scheduled processors with sufficient detail.

Simics: ``Reliable performance estimates require simulation of the full system''
        Their simulator includes device models accurate enough to run unmodified operating system kernels, firmware and device driver.
        They simulate a network of heterogeneous computers.
        Commercial product
        Two dimensional classification:
          scope (what is modeled),
          abstraction level (level at which the system is modeled)
            two perspectives: functional, timing (performance)

SimpleScalar:
  Hardware development is often accelerated by software simulations of the hardware under development.
  Basic idea: Implement a software model of the hardware, stress then with appropriate workload
  Gains: The availability of the simulation model is much faster than building the hardware. Thus the hardware can be tested faster. Shorter time to market, cut down development costs.
  Simulation modeling is driven by three factors: performance, flexibility, detail
    Performance measures the system's ability to process workload
    Flexibility reflects the ability to modify the models, enabling measurements of different designs
    Detail means the level of abstraction the model captures from the actual system
    -> Getting all three is hard, practically impossible.
    -> trade offs


\section{Model representation}
In PSE, the system under study is modeled as a resource reservation based mechanism \fixme{resource network?} \cite{Menasce:1994:CPP:174466}. The complete model consists of three main components: resource provision model, resource usage model and workload model. Each of the components are presented as directed graphs, which guide the simulation. The components are described in more detail in the following subchapters.

The model representations are built by using graphical user interfaces (workload editor, wle; task graph editor, tge; sequence chart editor, sce; resource network editor, rne). Each of the models are presented as directed graphs and the graph is saved as a text file.


 it presents the mechanisms for the needed to build a PSE model of a stream processing system consisting of a network processing hardware running a datapath (OpenEM) software.


PSE editor tools are used to present a system in PSE presentation format.

TRAVERSE

The events, generated by the workload model, traverse by the rules defined in the resource usage model consuming the resources defined in the resource provision model.

Workload model generates events, which consume the resources, defined in the resource provision model, according to the rules defined in the resource usage model.


\subsection{Resource provision model}
The resource provision model describes the resources, typically the hardware, of the system. The graph nodes present resources. Arcs present the possible usage order of the resources. The resources are consumed by the jobs \fixme{make the terminology constistent throughout the text} generated by the workload model.

A resource can be either active or passive. Active resources provide service and introduce delay to the events' using them. An example of active resource is processor core, which can serve certain amount of processing cycles per unit time. Passive resources do not induce delay to the jobs, but their possession is required to access certain other resources. Locks are an example of passive resource.

The resources can be limited in quantity.

Resources are represented as entities with attributes that describe ...

\subsection{Resource usage model}

The resource usage model is presented either as a task graph or a sequence chart

The resource usage model can viewed as a software, which guides the jobs' resource usage.

\subsection{Workload model}
The event spawn rate can be constant or random (specified for example with probability distribution).

When an event is spawned, it progresses through the resource provision model triggering the resource usages. -> gets delayed.

\section{Simulation and monitoring}
PSE provides a built-in discrete event simulator engine, RNS (Resource Network Simulator).

PSE models are simulated by Resource Network Simulator, RNS, which is a discrete event simulator engine.

The RNS simulator engine

% - Events from workload model
%   - Workload model creates new events with the event spawn rate described by the user

% - Each event starts by ``entering'' the resource usage model
%   - e.g. calls the function bound to the resource usage model that the user has provided as the JOB-parameter.

% - resource usage model is a set of functions generated by the tgc-compiler from the user created task graph file.
% - resource usage model calls the functions provided by the RNS api -> service delays

% - Active resources call RNS_demand_device or RNS_use_device, which induces a delay
%   - The resource reserved for the time the process is delayed

% - Passive resources reserve the resources using RNS_reserve_resource function
%   - Release happens when the event arrives at the function corresponding the release node specified in the resource usage network
%   - Unlike in the active resource usage, the resource is held by the event for a time unknown at the reserve time

\lstinputlisting[caption=RNS\_demand\_device,
                 label=RNS_demand_device]{listings/RNS_demand_device.c}
\lstinputlisting[caption=RNS\_use\_device,
                 label=RNS_use_device]{listings/RNS_use_device.c}
\lstinputlisting[caption=RNS\_reserve\_resource,
                 label=RNS_reserve_resource]{listings/RNS_reserve_resource.c}
\lstinputlisting[caption=RNS\_delay\_process,
                 label=RNS_delay_process]{listings/RNS_delay_process.c}
\lstinputlisting[caption=RNS\_release\_resource,
                 label=RNS_release_resource]{listings/RNS_release_resource.c}

\subsection{Probing}
The probes can be used to capture every state change of the system.

\subsection{Running the Simulation}
\subsection{Data pre-processing}


%%% Local Variables:
%%% mode: latex
%%% TeX-master: "thesis-hartikainen"
%%% End:
