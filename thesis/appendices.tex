\chapter{First appendix}
\label{chapter:first-appendix}
\lstinputlisting[caption=RNS\_use\_device,
                 label=lst:RNS-use-device]{listings/RNS_use_device.c}

RNS\_use\_device in~\ref{lst:RNS-use-device} reserves the resource, delays the process (i.e. the task) and releases the resource for other processes.

\lstinputlisting[caption=RNS\_demand\_device,
                 label=lst:RNS-demand-device]{listings/RNS_demand_device.c}

RNS\_demand\_device routine in~\ref{lst:RNS-demand-device} is a simple wrapper routine, that converts the demanded service amount (service\_amount) into corresponding service time, based on the device entity speed (d-$\textgreater$speed). It then calls the RNS\_use\_device routine with the resulting service time.

\lstinputlisting[caption=RNS\_reserve\_resource,
                 label=lst:RNS-reserve-resource]{listings/RNS_reserve_resource.c}

\ref{lst:RNS-reserve-resource} summarizes the RNS\_reserve\_resource routine. RNS\_reserve\_resource calls the reserve function bound to the resource entity as explained in section \ref{TODO}. The reserve function assigns the task either in the resource's processing queue or waiting queue.

If the reserve function assigns the task to the waiting queue, the thread yields the execution to the scheduler.

\lstinputlisting[caption=RNS\_delay\_process,
                 label=lst:RNS-delay-process]{listings/RNS_delay_process.c}
\lstinputlisting[caption=RNS\_release\_resource,
                 label=lst:RNS-release-resource]{listings/RNS_release_resource.c}

%%% Local Variables:
%%% mode: latex
%%% TeX-master: "thesis-hartikainen.tex"
%%% End:
