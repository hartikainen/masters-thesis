\chapter{Introduction}
\label{chapter:intro}
The proliferation of Internet devices and sensors have led to a situation where data are produced faster than can be transmitted, stored, or processed. Majority of the data is transmitted as packet streams over the packet switched networks. Along the way, various network devices, such as switches, routers, adapters, are used to forward and process the streams.

Packet processing devices face an enormous performance challenge. While the amount of data being transmitted is increasing, at the same time, the packet processing tasks, which should be done on-the-fly, have become more and more complex. In addition to the usual forwarding, the packet processing systems are responsible of various other functions, such as traffic management (shaping, timing, scheduling), security processing, and quality of service. At the same time, technology and customer requirements are rapidly changing, forcing the vendors to seek programmable solutions to achieve shorter development cycles and more revisions.

Network processing equipment can generally be divided into three categories: easily programmable general Central Processing Units (CPU), well-performing but hardwired Application-Specific Integrated Circuits (ASIC), and middle-ground Network Processing Units (NPU). The focus of this thesis is on the middle-ground NPU devices, typically built as a multiprocessor system-on-chip (MPSoC). MPSoCs integrate multiple heterogenous or homogenous functional units, such as processors, memory, circuits, and peripherals, on a single chip. The promise of MPSoC devices are their better performance,  functionality, and energy usage, due to the multiple interconnected specialized processors.

Network processing units are, in essence, queuing systems. Different components of the MPSoC devices, such as network interface cards, CPUs, and the global system scheduler, queue the packets in memory between the processing steps. Each component fetches the packets from the memory, or queues, based on certain rules that can be seen as queue disciplines.

The treatment of the packets flowing through the network processing equipment is called packet processing. Packet processing consists of series of functions performed, typically in parallel, on the packets. The functions are often implemented as separate software applications. Different parallel programming models, such as task-parallelism, are used to abstract the complexity of these devices, enabling efficient application development and portability. In the packet processing context, the task parallelism can be seen as each task consisting of an operation (code) done on a packet (data). Again, the task parallel programming frameworks can be seen as a set of queues abstracting the underlying hardware.

Packet processing systems are complex. To keep up with the ever increasing performance requirements, the behavior of the systems needs to be understood. This thesis investigates the use of measurement, modeling, and simulation methods for the use of packet processing performance analysis. We present a way to model complex hardware and software interactions of a modern MPSoC network processing systems and task-based parallel programming frameworks, using an extended resource networks concept. Resource networks are based on the queueing networks concept, often used for packet processing problems and other computing system modeling, thus being a natural way of describing MPSoC systems and task-based parallel programming applications.

The tool chosen for the modeling and simulation is an in-house simulator and modeling software, Performance Simulation Environment (PSE). We will instrument and measure the characteristics of a modern packet processing system, and build a resource network model using PSE's built-in editors. We extend PSE to support user definable queue disciplines, enabling more detailed models of the systems global packet scheduler. We present two demonstrative experiments, showing that the model and PSE extensions work as expected. The models are simulated using PSE's discrete event simulator engine.

The experiments demonstrate that, with extended custom queue disciplines and support for modeling task parallelism, resource network methodology is a viable approach for performance analysis of such systems.

\section{Problem statement}
Modern MPSoC based network processing units are complex parallel processing systems with complex hardware interactions. Understanding these systems is difficult, due to the architectural heterogeneity, complexity, and non-deterministic behavior of the typical multiprocessor system-on-chip (MPSoC) implementation. The non-deterministic behavior of the MPSoCs is a result of parallelism, communication delays, complex memory systems, and hardware and software scheduling mechanisms.

% These system interactions are difficult, sometimes even impossible, to reason about, and make the performance analysis non-trivial.

The nature of packet switched network data introduces strict performance requirements for the packet processing devices: the manipulation of the packet streams passing through the system must often be done on the on-the-fly, while at the same time, the data volumes are often huge and unpredictable.

The performance of the task parallel applications is heavily dependent on the scheduling mechanism of the system. To guarantee the performance of task parallel programming models, the scheduler has to efficiently map the functions and minimize the context switch overhead. MPSoC devices are inherently parallel, and the scheduling is often done globally at multiple points of the system.

% Important aspect of the performance of task parallel programming mod- els is how the tasks are scheduled on the cores. OpenEM features a dynamic scheduler, which means the execution order of tasks is decided in runtime. Another way of scheduling tasks would be static scheduling, where the exe- cution order of the tasks is determined at the time of compilation.


Queueing and resource networks are widely studied concepts, often used for packet processing problems and other computing system modeling. However, the complex hardware level interactions between the MPSoC components, dynamic workload, and task-based parallelism methods, make the traditional queue concepts insufficient. One way to address this problem, is to extend resource networks with user-definable queue disciplines allowing global view of the system, and methods for modeling parallelism.

The research problem of this thesis is: How to analyze the performance of packet processing systems. Or more precisely: how to extend the resource network concept to support more accurate models of hardware accelerated manycore systems, and on the other hand, how to support modeling of packet processing task-based programming models.

\section{Contributions}
We will present a method for modeling and simulation of the hardware accelerated manycore packet processing systems. We extend the in-house discrete event simulator, Performance Simulation Environment (PSE), to enable use of customized queue discipline algorithms. We will also model a modern network processing unit using the tools provided by PSE and simulate the model by PSE's discrete event simulation engine. The major contributions of our work can be summarized as:

% The main contributions of our work are three-fold. First, we have instrumented and measured the characteristics of a customized network processing system. Secondly, we extended the toolset of an existing in-house simulator software to enable modeling customized queue disciplines, allowing. Lastly, we built a simulation model based of the network processing unit and ran experiments to validate the proper model behavior.


\begin{itemize}

\item Instrumenting and measuring the characteristics of a modern network processing unit

\item Extending an existing in-house modeling and simulation framework, Performance Simulation Environment (PSE), to support user-definable queue disciplines through a plugin interface %more complex hardware scheduling functions

% \item Refactoring the existing PSE functionalities and fixing several software bugs that made critical parts of the simulator unusable

\item Designing and building a PSE model of a MPSoC based packet processing system

% \item The results were presented to the industry partners at the ParallaX group's fall assembly, working as a concrete example of the Embedded Systems Group's work in progress

\item Proof of concept experiment demonstrating performance analysis of the implemented hardware model and task-based parallel application models

% \item The PSE extensions and the network processing unit model has been used as the building blocks for the experiment of further research on Fog computing
\end{itemize}

Our first contribution includes setting up an environment for measuring the memory and communication delays of a packet processing system. The measurements are used to gain in-depth understanding of the system interaction of such systems, and to further determine the correct abstraction level for our performance analysis models. The measurements consisted of executing several microbenchmarks using low level hardware application programming interfaces. The automatization scripts and test programs are documented and available to be used in further experiments on similar systems.

The extension of PSE includes a new plugin code mechanism, which allows customized queue discipline algorithms to be defined by external C-code. The design and implementation affected nearly all parts of the underlying simulation framework. In addition, PSE's memory usage has been improved to enable simulation of larger systems and workloads. Together the extension and other improvements consisted of roughly 700 edited and 600 new lines of code. These features are documented and available to be further utilized by PSE users.

Our third major contribution is the implementation of simulation model of a multi-blade network processing system. The models are based on our insights gained from the measurement and studies of a reference system and packet processing applications written on task-based programming frameworks. They consists of decoupled workload, software, and hardware models, highlighting the ability to decouple the different functional parts of the system, and enabling modularization and further reuse. The packet processing software applications are modeled using event based programming paradigms. The global packet scheduler, which is the key for modeling task-based applications, is modeled using the implemented plugin code mechanism.

Finally, we have built a proof-of-concept experiment to validate the models and implemted PSE extensions. Further, it demonstrates how the implemented models can be used for performance analysis of a network processing system. With the decoupled models, we are able to easily adjust the software parameters, affecting the sought performance metrics, such as latency and throughput, of the system.

\section{Structure}
Chapter~\ref{chapter:computing-trends} presents an overview to the context of this thesis. It motivates the performance analysis of packet processing systems. We will describe the reasons that have led the IT industry to widely adopt paradigms called cloud and fog computing. Further, we describe the cloud and fog computing together with the relevant technologies, such as virtualization and software defined networking, enabling these paradigms.

Next, in Chapter~\ref{chapter:packet-processing-systems}, we explain the concepts needed to understand the functionality of modern packet processing hardware and software, and their relation to queuing theory. We will present a more detailed view on task-based programming framework, called Open Event-Machine, and example of pipelined hardware architecture of network processing systems. The chapter finishes with characteristic measurements of a network processing system to gather in-depth understanding of such systems system.

In Chapter~\ref{chapter:system-performance-analysis}, we will present the basic concepts of performance analysis, modeling, and simulation. We will begin by presenting different evaluation techniques and performance metrics, further defining the components of system under a performance study. Then, we present the basic concepts of modeling, and queuing and resource networks, to underline their usage in traditional packet processing systems performance analysis. Finally, we will describe the simulation model and monitoring, with a short survey of the existing simulation software.

Chapter~\ref{chapter:performance-simulation-environment} presents the simulation tool, Performance Simulation Environment (PSE), used in our work. The chapter begins with an overview of the PSE's toolset. After that, we will describe the three main components of a PSE model. Finally, we will present the built-in discrete event simulator of PSE.

Chapter~\ref{chapter:mechanism-for-extended-queue-disciplines} presents the implemented plugin code mechanism for Performance Simulation Environment. The extensions enable modeling of user-defined queue disciplines written in C-code, and is our attempt to address the lack of global queue scheduling, which is required to use PSE for more detailed modeling of hardware scheduled manycore systems.

The example model of a packet processing system is presented in Chapter~\ref{chapter:modeling-a-packet-processing-system}. We will first describe the main characteristics of the model, and further describe a more detailed view of the global packet scheduler functionality.

The demonstrative experiments are presented in Chapter~\ref{chapter:demonstrative-experiments}. We will describe the two experiment setups used, and their measurement results. Finally, we will analyze the experiments results and discuss the experiment.

The conclusion of our work are presented in the Chapter~\ref{chapter:discussion}. It discusses the challenges, discoveries, and the related and future work within the topic of the thesis, and finally concludes it.

%%% Local Variables:
%%% mode: latex
%%% TeX-master: "thesis-hartikainen"
%%% End:
