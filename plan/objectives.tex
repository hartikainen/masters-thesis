The objectives of this thesis is to understand the allocation of computing resources in a streaming multi-blade computing setup. This objective includes an implementation(s) of efficient scheduling/load-balancing for multi-blade processing, as well as the analysis of those implementation(s). Hopefully we can answer the question, how should the multi-blade load-balancing be implemented? Should it be implemented on top of OpenMP/OpenEM/MPI?

% The event based parallelism seems promising, and thus the question is, how to manage the load-balancing between multiple blades? Should the OpenMP/OpenEM be extended, or implement something else?

% One of the objects is to understand load-balancing of multiple blades. Events will be a part of this. ``Työkaluina eventit, miten balansoidaan multibladelle?'' -> ``ei riitä että miten Cavium toimii''

% ``Asian muotoilu on kriittinen juttu, essential and not easy''

% The functions of Cavium itself are not so interesting, but how to do load-balancing on multiple blades.

% Qualitative evaluation vs quantitative evaluation
% The experiments will be done using BCN/Cavium, and thus the focus will be on the quantitative side.
% ``cavium hard, black box''

% ``This section should be like the abstract, not quite as formal though''

TODO:
\begin{itemize}
\item Tavoitteet vielä melko epäselvät, ainakin vaikeasti muotoiltavissa.
\item Kuten BG:ssä, mikä on se oikea ongelma?
\item Onko dynaaminen/staattinen analyysi vain väline näiden tulosten evaluointiin, vai onko analyysissä itsessään jotain kysymysmerkkejä joihin halutaan vastauksia??
\item Ilmeisesti pääkysymys on, että miten load-balancing toteutetaan multi-bladelle, toteutetaanko suoraan OpenEM:ään/OpenMP:hen vai jollain muulla tavalla?
\end{itemize}


%%% Local Variables:
%%% mode: latex
%%% TeX-master: "thesis_plan_222956"
%%% End:
