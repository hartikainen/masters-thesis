The amount of data being transmitted and processed in the world is growing rapidly, and the restrictions, mainly the increasing energy consumption, in the microprocessor evolution has led in the development of new type of Multiprocessor System-on-Chip devices (MPSoC).

While these MPSoC devices seem promising in terms of the needed increase in computational performance, they also bring out new challenges in the software development. The traditionally used parallel programming models, that are based on thread parallelism, don't meet the needs of processing the heterogeneous stream-like data.

Several different programming models/frameworks have been proposed to allow parallel software development. It is still unclear, however, how to efficiently allocate the computing resources in these models.

Also, the analysis of parallel programs is a tedious task, partly due to the non-deterministic behaviour of the parallel hardware, the complexity in the memory accesses and the delays from the communication.

TODO:
\begin{itemize}
\item Mikä on oikeastaan on se yleinen ongelma täällä? Ilmeisesti multiblade juttu itsessään on melko selkeää, samoin kun eventtipohjainen parallelismi. Onko stream-processing, yhdistettynä näihin kahteen, se joka tuo ongelman tähän hommaan?
\item Miten analyysi liittyy näihin asioihin? Onko haasteet analyysin tekemisessä itsessään jo ongelma vai johtaako ne vain epäsuorasti ongelmiin esim. eri menetelmiä verrattaessa??
\end{itemize}

%%% Local Variables:
%%% mode: latex
%%% TeX-master: "thesis_plan_222956"
%%% End:
