Parallel computation and heterogeneous MPSoC devices seem promising in terms of the needed increase in today's computational performance. However, they also bring out new challenges in the software development and requirements for the programming models.

The scalability of the parallel computation itself, especially on heterogeneous hardware, is a hard problem due to the non-determinism of parallel computation, communication delays and complicated memory management. Data processed in stream computation is often dynamic, that is, the volume of the data and the computation requirements can change at any point of the computation.

Hardware threads are static, meaning that they cannot adapt to the dynamic changes of the workload. Also, the hardware threads cannot be migrated between the computing nodes, and thus binding the computation to the hardware threads forces the computation to be done on the specific computing node. This is why the traditionally used parallel programming models, that are based on thread parallelism, don't meet needs of streaming computing of dynamically changing workflow.

Traditional computing clusters and inter-node computations can be virtualized by using message passing abstractions such as MPI. However the real time nature of the streaming computation brings on problems to this kind of inter-node computation. The research problem of this thesis is the balancing and transferring of the dynamic workload between multiple stream computing nodes.

The thesis is done under the Embedded Systems Group in Aalto University with connections to the ParallaX research project. The actual starting point for this thesis is defined by the recent bachelor's theses (Kiljunen, Teemaa) and master's theses (Hanhirova, Saarinen, Rasa), written for the Embedded Systems Group.

%%% Local Variables:
%%% mode: latex
%%% TeX-master: "thesis_plan_222956"
%%% End:
